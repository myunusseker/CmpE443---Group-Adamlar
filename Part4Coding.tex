\documentclass{article}
\usepackage{array}
\usepackage[a4paper, total={6in, 8in}]{geometry}
\newcolumntype{L}[1]{>{\raggedright\let\newline\\\arraybackslash\hspace{0pt}}m{#1}}
\newcolumntype{C}[1]{>{\centering\let\newline\\\arraybackslash\hspace{0pt}}m{#1}}

\begin{document}
{\huge\textbf {Module: Lap Counter}}

\begin{tabular}{| C{4cm} | C{4cm} | C{4cm} | C{2cm} |}
\hline
\textbf{Function Name} &\textbf{Function Definition}  & \textbf{Objective} &\textbf{WCET}\linebreak(simulation)\\
\hline
Ultrasonic\_Init() & Initialization of the Ultrasonic sensor. Functions of the trigger
and echo pins connected to the LPC board are defined in the IOCON registers& There are 2 data pins to
communicate with the ultrasonic sensor. Enabling the corresponding pins in the main board, their directions'
(input or output) is set for echo and trigger pins. So that, the board is able to trigger a calculation
of the distance and get the incoming signals with echo pin.& \\
\hline
Ultrasonic\_Trigger\_Timer\_Init()&Trigger timer initializer. The output for the trigger pin is
 initialized in this function & Timer 2 is used to synchronize the value(either HIGH or LOW) to the Ultrasonic
 sensor. To initialized it, this function is used. First the power is enabled to this section with PCONP Register
 , then counters (Timer counters and Prescale counters) and match values are set appropriately. Last, the
 function on a match("toggle" in this case) is set. & \\
\hline
LCD\_init()&Initialization of the LCD and its pins.& LCD is initialized by setting its pins' functionalities on the board,
and sending correct values to initialize the external driver, ( like 0x03, 0x03,0x03,0x02)&\\
\hline
\end{tabular}
\begin{tabular}{| C{4cm} | C{4cm} | C{4cm} | C{2cm} |}
\hline
Ultrasonic\_Capture\_Timer\_Init()& Echo timer initializer for the echo signal of the ultrasonic sensor.&
First the using PCONP register, the timer is powered on. Then its match registers and timer and prescale timer registers
are given the appropriate functionalities. This timer is used to count the elapsed time from the transmission of the
signals to the echo of the sent signals. So that, the distance of the object could be detected.&\\
\hline
Serial\_init()&Initialization of the UART 0 & Using UART(Serial communication), the debug information of distance of the detected object
and the lap count will be sent to the computer. &\\
\hline
TU()&Task for Ultrasonic, calculates the distance&The distance is calculated detected by ultrasonic sensor.
This distance is used to detect whether the car is in the range or not.&\\
\hline
TC()& Task for the lap counter & After getting the distance detected, this function checks whether the
distance is smaller than the threshold, if it is, then the lap count is incremented by one. &\\
\hline
\end{tabular}
\begin{tabular}{| C{4cm} | C{4cm} | C{4cm} | C{2cm} |}
\hline
TSD()&Task for system diagnosis, includes sending distance detected and the lap count data from UART
to PC & In order to debug the code, we are using this function which will be printing the value of
distance measured and the lap counted.&\\
\hline
TDi()&This function sets the cursor to the appropriate position, then clears the display starting from the
cursor, then writes the lap count into the LCD screen.& To meet the requirements specified,
this function is used. It basically, writes the lap count into the LCD screen.&\\
\hline
\end{tabular}
{\huge\textbf {Module: Getting Speed Data from Input Device}}
\end{document}
