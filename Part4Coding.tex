\documentclass{article}
\usepackage{array}
\usepackage[a4paper, total={6in, 8in}]{geometry}
\newcolumntype{L}[1]{>{\raggedright\let\newline\\\arraybackslash\hspace{0pt}}m{#1}}
\newcolumntype{C}[1]{>{\centering\let\newline\\\arraybackslash\hspace{0pt}}m{#1}}

\begin{document}
{\huge\textbf {Module: Lap Counter}}

\begin{tabular}{| C{4cm} | C{4cm} | C{4cm} | C{2cm} |}
\hline
\textbf{Function Name} &\textbf{Function Definition}  & \textbf{Objective} &\textbf{WCET}\linebreak(simulation)\\
\hline
Ultrasonic\_Init() & Initialization of the Ultrasonic sensor. Functions of the trigger
and echo pins connected to the LPC board are defined in the IOCON registers& There are 2 data pins to
communicate with the ultrasonic sensor. Enabling the corresponding pins in the main board, their directions'
(input or output) is set for echo and trigger pins. So that, the board is able to trigger a calculation
of the distance and get the incoming signals with echo pin.& \\
\hline
Ultrasonic\_Trigger\_Timer\_Init()&Trigger timer initializer. The output for the trigger pin is
 initialized in this function & Timer 2 is used to synchronize the value(either HIGH or LOW) to the Ultrasonic
 sensor. To initialized it, this function is used. First the power is enabled to this section with PCONP Register
 , then counters (Timer counters and Prescale counters) and match values are set appropriately. Last, the
 function on a match("toggle" in this case) is set. & \\
\hline
Ultrasonic\_Capture\_Timer\_Init()&&&\\


\hline
\end{tabular}
\end{document}
